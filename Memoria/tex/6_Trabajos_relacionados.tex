\capitulo{6}{Trabajos relacionados}

 En este apartado, se van a comentar algunas de las referencias, trabajos y aplicaciones que he tenido en cuenta para desarrollar esta aplicación.

 \subsection{Proyectos}
 Inicialmente, busqué proyectos que utilizarán la tecnología de imageAI para investigar y aprender más del funcionamiento acerca de esta herramienta.
 Tomé de referencia de partida un Trabajo de fin de grado de la Universidad de Sevilla llamado 'Detección de objetos en imágenes utilizando técnicas de aprendizaje profundo (Deep Learning)' \cite{tfgSevilla:latex}. \\
 En este trabajo de fin de grado, se ha utilizado ImageAI para el procesamiento de vídeos deportivos. Analiza los distintos algoritmos que proporciona ImageAI(YoloV3, Retinanet, TinyYOLOv3) y saca conclusiones a través de los resultados generados. \\

 Mi proyecto utiliza imageAI y lo implementa en una aplicación para que un usuario común pueda utilizar está herramienta sin ninguna complicación. Además, a diferencia de este trabajo de fin de grado, FutboStats conforma una aplicación completa de fútbol y estadísticas que brinda al usuario datos en tiempo y herramientas novedosas que se pueden aplicar en el fútbol como es ImageAI.

 \subsection{Aplicaciones}
 Para desarrollar mi aplicación, tomé como referencia la aplicación móvil llamada 'One Football' \cite{oneFootball}. \\
 OneFootball es una popular aplicación móvil dedicada a ofrecer noticias, estadísticas y contenido multimedia sobre fútbol. Ofrece una amplia cobertura de ligas y competiciones de todo el mundo, incluyendo partidos en vivo, resultados, tablas de clasificaciones y actualizaciones de los equipos y jugadores.

Mi proyecto además de ser una web aplicación web implementa gráficos novedosos diseñados y creados en Python para representar estadísticas de partidos. Además, integra un modelo de goles esperados y de puntos esperados para informar al usuario de estadísticas más completas acerca de un equipo o una liga. \\

\definecolor{lightgreen}{RGB}{144, 238, 144}
\definecolor{lightred}{RGB}{255, 182, 193}

\begin{table}[h]
\centering
\begin{tabular}{|l|c|c|c|}
\hline
\textbf{} & \textbf{TFG US} & \textbf{One Football} & \textbf{FutboStats} \\ \hline
\textbf{Uso de ImageAI} & \cellcolor{lightgreen} Sí & \cellcolor{lightred} No & \cellcolor{lightgreen} Sí \\ \hline
\textbf{ImageAI en web} & \cellcolor{lightred} No & \cellcolor{lightred} No & \cellcolor{lightgreen} Sí \\ \hline
\textbf{Consulta de datos} & \cellcolor{lightred} No & \cellcolor{lightgreen} Sí & \cellcolor{lightgreen} Sí \\ \hline
\textbf{Goles esperados (xG)} & \cellcolor{lightred} No & \cellcolor{lightred} No & \cellcolor{lightgreen} Sí \\ \hline
\textbf{Puntos esperados (xPoints)} & \cellcolor{lightred} No & \cellcolor{lightred} No & \cellcolor{lightgreen} Sí \\ \hline
\textbf{Estadísticas con Python} & \cellcolor{lightred} No & \cellcolor{lightred} No & \cellcolor{lightgreen} Sí \\ \hline
\textbf{Aplicación web} & \cellcolor{lightred} No & \cellcolor{lightgreen} Sí & \cellcolor{lightgreen} Sí \\ \hline
\textbf{Adaptación a móvil} & \cellcolor{lightred} No & \cellcolor{lightgreen} Sí & \cellcolor{lightgreen} Sí \\ \hline
\end{tabular}
\caption{Comparativa de funcionalidades entre el TFG de la Universidad de Sevilla, One Football y FutboStats}
\label{table:comparativa}
\end{table}





