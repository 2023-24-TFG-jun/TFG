\capitulo{7}{Conclusiones y Líneas de trabajo futuras}

En este apartado se exponen las conclusiones del proyecto y las líneas de trabajo futuras mediante las cuales puede ir escalando el proyecto.

\subsection{Conclusiones}

Tras finalizar el desarrollo del proyecto propuesto he sacado las siguientes conclusiones:
\begin{itemize}
    \item He cumplido el objetivo inicial del proyecto gratamente. Se ha desarrollado una aplicación web desde cero e interactiva para cualquier usuario que desee consultar datos y estadísticas variadas sobre fútbol.
    \item Utilizar Angular para el desarrollo \textit{front-end} ha supuesto ventajas e inconvenientes. No han surgido problemas con las dependencias en Angular utilizadas y ha supuesto, es un \textit{framework} escalable bien estructurado, se pueden reutilizar los componentes creado fácilmente. Algunas de sus desventajas se basan en su curva de aprendizaje al iniciar el proyecto, ya que, ha supuesto un reto aprender Angular en un periodo reducido de tiempo, además de las extensas funcionalidad que contiene el \textit{framework}.
    \item El proyecto ha abarcado conocimientos sobre librerías de Python utilizadas durante el grado como pueden ser Pandas, Matplotlib, SciPy, modelos de regresión, etc.
    \item El proyecto ha abarcado una gran cantidad de conocimientos que desconocía en el inicio. Ha supuesto un reto aprender Angular y aplicarlo en un proyecto real tan pronto, pero gracias al esfuerzo y dedicación he conseguido desarrollar con éxito una aplicación usable por cualquier usuario.
    \item En este proyecto he utilizado un gran número de herramientas y tecnologías como ImageAI que me han ayudado a implementar funcionalidades nuevas a la aplicación y con ello mejorar el producto final que se ofrece al usuario. 
    \item Se ha desarrollado el proyecto en el plazo establecido de manera satisfactoria.
    \item Gracias a esta aplicación he aprendido a utilizar la documentación de una API y como aplicarlo en un proyecto real. Además, he aprendido a utilizar la herramienta Postman, la cual es frecuentemente usada en el mundo laboral para realizar llamadas reales a los endpoints de una API y comprobar si la respuesta es la esperada o no. Esto suponía ahorrar tiempo a la hora de implementar correctamente las funcionalidades en la aplicación.
\end{itemize}

\subsection{Líneas de trabajo futuras}
A continuación, se van a detallar algunas de las mejoras y líneas de trabajo futuras que tendrá este proyecto:
\begin{itemize}
    \item Implementación de nuevas funcionalidades con API-FOOTBALL. Debido a la gran cantidad de datos en tiempo real que proporciona API-Football, se implementarán nuevas funcionalidades en la aplicación que mejoran la experiencia al usuario y enriquezcan la aplicación.
    \item Creación de un pequeño buzón de sugerencias al usuario para que nos haga llegar un \textit{feedback} y como podríamos mejorar la aplicación.
    \item Actualizar las estadísticas en la parte de predicciones regularmente con datos de partidos y competiciones recientes.
    \item Internacionalizar la aplicación implementando traducciones para que sea posible utilizar la aplicación en varios idiomas.
\end{itemize}

Estas son algunas de las mejoras que se integrarán en la aplicación en el futuro, haciendo crecer a la aplicación para que llegue a un mayor número de usuarios.
