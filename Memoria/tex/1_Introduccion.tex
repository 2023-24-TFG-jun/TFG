\capitulo{1}{Introducción}

En la actualidad, el deporte ha sido uno de los campos que más ha aprovechado las innovaciones digitales para transformar cómo se analizan y se entienden los partidos o juegos. La integración de la tecnología en el deporte no solo ha mejorado el rendimiento de los atletas y equipos, sino que también ha revolucionado la experiencia de los aficionados y profesionales del sector. Este Trabajo de Fin de Grado presenta el desarrollo de una aplicación web deportiva centrada en el fútbol, que busca servir como herramienta para el análisis estadístico y la predicción de resultados en este deporte. \\

La aplicación aporta a los usuarios la posibilidad de ver estadísticas detalladas y buscar partidos según diferentes criterios, ofreciendo detalles sobre los equipos y sus posiciones en distintas ligas. Al emplear una API externa, la aplicación proporciona información actualizada y exacta que resulta imprescindible para cualquier estudio deportivo. También cuenta con una parte de predicciones que, usando modelos matemáticos, calcula los goles esperados de cada equipo, ofreciendo predicciones sobre los puntos que podrían lograr al final de la temporada. \\

La métrica innovadora de los goles esperados evalúa la posibilidad de que una oportunidad de gol se convierta en gol considerando factores como la posición del disparo y la formación defensiva. Este número no solo predice el rendimiento futuro, sino que también proporciona una perspectiva más completa del juego, lo que permite a entrenadores y analistas entender mejor las tácticas y el desempeño de los equipos. \\

La aplicación implementa como característica novedosa el uso de \textit{ImageAI} (biblioteca de inteligencia artificial que permite realizar detección de objetos en imágenes y vídeos) para el análisis de vídeos de partidos de fútbol. Mediante técnicas de inteligencia artificial, se realiza un seguimiento detallado de los movimientos de los jugadores en el campo, proporcionando visualizaciones que mejoran el análisis táctico y la comprensión de las dinámicas del juego. Esta tecnología no solo enriquece la estrategia y la preparación de los equipos, sino que también mejora la narrativa para los seguidores del deporte, permitiendo una apreciación más rica y detallada de las habilidades y estrategias en juego. \\

En conclusión, esta aplicación no solo es un reflejo de cómo la tecnología puede ser aplicada para enriquecer el análisis deportivo, sino que también es una herramienta que podría influir significativamente en la toma de decisiones en el fútbol profesional. Al combinar estadísticas avanzadas, predicciones precisas y análisis visual de movimientos, puede llevarnos a una nueva forma de entender el fútbol, marcando un punto de inflexión para entrenadores, jugadores, analistas y, en última instancia, para los aficionados. \\


\section{Estructura de la memoria}

La memoria sigue la siguiente estructura:

\begin{itemize}
    \item \textbf{Introducción:} Breve descripción de los contenidos del proyecto y de la aplicación sin entrar en detalles técnicos. Estructura de la memoria y listado de los materiales adjuntos.
    \item \textbf{Objetivos del proyecto:} Descripción de los objetivos que busca el proyecto.
    \item \textbf{Conceptos teóricos:} Breve explicación de los conceptos teóricos importantes para la comprensión de la solución propuesta.
    \item \textbf{Técnicas y herramientas:} Descripción de las técnicas metodológicas y herramientas utilizadas para la organización y desarrollo del proyecto.
    \item \textbf{Aspectos relevantes del desarrollo:} Exposición de los problemas y soluciones que se llevaron a cabo en el transcurso de la realización del proyecto.
    \item \textbf{Trabajos relacionados:} Proyectos y aplicaciones deportivas que me han servido de inspiración para desarrollar el proyecto.
    \item \textbf{Conclusiones y líneas de trabajo futuras:} Conclusiones obtenidas tras la realización del proyecto y posibilidades de mejora.
\end{itemize}

\section{Materiales adjuntos entregados}

Los materiales que se adjuntan junto con la memoria en el usb son:

\begin{itemize}
    \item Aplicación Angular para el desarrollo del front-end.
    \item Aplicación Flask para el desarrollo de los algoritmos de predicciones, uso de imageAI, generación de gráficos.
    \item Vídeos de uso de la aplicación.
    \item Anexos del proyecto.
    \item Memoria del proyecto.
    \item Máquina virtual para ejecutar el proyecto en local.
    \item Fichero README con claves de la máquina virtual como el usuario, correo de Google para probar las funcionalidades que requieren iniciar sesión.
\end{itemize}

Otros recursos accesibles a través de internet:

\begin{itemize}
    \item Página web del proyecto: \href{https://futbostats.netlify.app/}{https://futbostats.netlify.app/}
    \item Repositorio del proyecto en GitHub: \href{https://github.com/MiguelExtremo/TFG}{https://github.com/MiguelExtremo/TFG} 
\end{itemize}
