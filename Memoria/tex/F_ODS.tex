\apendice{Anexo de sostenibilización curricular}

\section{Introducción}
FutboStats es una aplicación web para la consulta de datos de fútbol, estadísticas y procesado de vídeos. Ha permitido abordar varios aspectos clave de la sostenibilidad. A continuación, se detallan las competencias de sostenibilidad adquiridas y aplicadas en este proyecto, en relación con los objetivos de desarrollo sostenible (ODS).

\subsection{Salud y bienestar (ODS 3)}
FutboStats promueve la actividad física y el bienestar al proporcionar análisis detallados y visualización de datos deportivos. Esta herramienta puede ser utilizada por entrenadores, jugadores y aficionados para mejorar el rendimiento deportivo y fomentar estilos de vida saludables.

\subsection{Educación de calidad (ODS 4)}
Durante el desarrollo de FutboStats, se han adquirido y aplicado conocimientos avanzados en análisis de datos, programación y desarrollo web. Estos conocimientos no solo mejoran las competencias técnicas del alumnado, sino que también pueden ser compartidos y enseñados a otros estudiantes, contribuyendo así a una educación de calidad y al desarrollo profesional.
El proceso de desarrollo del proyecto ha involucrado el aprendizaje de lenguajes de programación como Python, TypeScript, HTML, CSS, así como el uso de bibliotecas y herramientas específicas como StatsBomb e imageAI. Además, la experiencia en el manejo de datos deportivos y la creación de visualizaciones complejas proporciona un conjunto de habilidades aplicables en diversas áreas tecnológicas y científicas. Este conocimiento puede ser transmitido a otros estudiantes ampliando el impacto educativo del proyecto.

\subsection{Energía Asequible y No Contaminante (ODS 7)}
El desarrollo de aplicaciones eficientes en términos de energía y recursos computacionales contribuye a la sostenibilidad ambiental. FutboStats ha sido desarrollado teniendo en cuenta la optimización del uso de recursos, lo cual es un paso hacia la creación de soluciones tecnológicas más sostenibles. La eficiencia energética en el desarrollo de software implica minimizar el uso de recursos computacionales y reducir el consumo de energía en servidores y dispositivos de los usuarios. Esto se logra mediante la optimización del código y la implementación de prácticas de programación sostenible. Al reducir la huella de carbono de las aplicaciones tecnológicas, FutboStats contribuye a la protección del medio ambiente y promueve el uso responsable de la energía.

\subsection{Conclusiones}
En resumen, el desarrollo de FutboStats no solo ha permitido la adquisición de competencias técnicas avanzadas, sino que también ha promovido la reflexión y aplicación de principios de sostenibilidad en diversas áreas. Este proyecto demuestra cómo la tecnología y el análisis de datos pueden contribuir a un desarrollo más sostenible y equitativo, alineado con los objetivos de desarrollo sostenible. \\
FutboStats, al abordar múltiples ODS, resalta la importancia de integrar la sostenibilidad en el desarrollo tecnológico. La promoción de la salud y el bienestar, la mejora de la educación y la optimización del uso de energía  son solo algunas de las formas en que este proyecto contribuye a un futuro más sostenible. \\
Continuar con el desarrollo de FutboStats permite seguir explorando nuevas formas de integrar la sostenibilidad en el deporte y la tecnología, asegurando que estas prácticas se conviertan en una parte importante de la industria.