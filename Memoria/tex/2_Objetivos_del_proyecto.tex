\capitulo{2}{Objetivos del proyecto}

A continuación,  voy a detallar en tres tipos los objetivos que se persiguen con el proyecto y que han motivado la realización del mismo.

\section{Objetivos generales}

\begin{itemize}
    \item Desarrollar una aplicación web de fútbol y estadísticas que permita a los usuarios consultar datos y estadísticas de forma interactiva.
    \item Facilitar al usuario la interpretación de los algoritmos y estadísticas mediante gráficas y tablas.
    \item Implementar en una aplicación web un sistema de tracking para visualizar el movimiento de los jugadores en el campo.
    \item Proporcionar al usuario información actualizada sobre estadísticas.
    \item Desarrollar una interfaz de usuario intuitiva.
\end{itemize}

\section{Objetivos técnicos}

\begin{itemize}
    \item Utilizar el \textit{framework Angular} para estructurar un front-end modular y mantenible, que permita que el proyecto crezca y mejore en el futuro de forma sencilla. Además, adaptar la aplicación haciéndola responsiva para mejorar la compatibilidad de la aplicación web con móviles.
    \item Desarrollar servicios de back-end usando \textit{Flask} para manejar peticiones \textit{REST} desde el front-end.
     \item Implementar protección de rutas mediante el uso de \textit{guards} en Angular para proteger las rutas a usuarios en función de roles.
    \item Integrar análisis de vídeos deportivos implementando una herramienta de tracking para observar cómo los jugadores se mueven en el campo.
    \item Integrar una API externa que me proporcione datos actualizados de partidos, equipos y ligas. Manejar errores para garantizar una buena experiencia en el usuario.
    \item  Utilizar los datos de \textit{StatsBomb} para incluir métricas avanzadas como los goles esperados (xG), generar mapas de calor, mapas de pases. Desarrollar algoritmos que generen predicciones detalladas sobre clasificaciones y equipos.
    \item Desplegar la aplicación para su puesta en producción.
    \item Utilizar un sistema de control de versiones como \textit{Git} para facilitar el seguimiento de los cambios en el código de la aplicación. Llevar a cabo buenas prácticas de desarrollo de software como revisiones de código para mantener un código limpio y coherente.
    \item Desarrollar pruebas unitarias para asegurar que el código funcione correctamente.
    \item Crear guías de instalación, configuración, uso para facilitar a los usuarios el uso del proyecto.
    
\end{itemize}

\section{Objetivos personales}
\begin{itemize}
    \item Introducirme al mundo del análisis deportivo y la estadística.
    \item Formarme en nuevos lenguajes de programación (HTML5, CSS3, TypeScript) y en el desarrollo de una aplicación web para navegador de escritorio y móvil.
    \item Explorar metodologías y herramientas utilizadas globalmente en el mundo laboral como Angular para el desarrollo \textit{front-end} y Flask para el desarrollo \textit{back-end}.
    \item Introducirme en tecnologías de \textit{tracking} de objetos o personas en vídeos con algoritmos de inteligencia artificial.
\end{itemize}
